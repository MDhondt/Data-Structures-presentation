\documentclass[usenames,dvipsnames]{beamer}
\usetheme{Copenhagen}
\useinnertheme{circles}
\useoutertheme{split}
\setbeamertemplate{itemize items}[triangle]
\beamertemplatenavigationsymbolsempty

\usepackage[english]{babel}
\usepackage[utf8]{inputenc}
\usepackage[T1]{fontenc}
\usepackage{lmodern}
\usepackage{textcomp}
\usepackage{tikz}
\usepackage{eso-pic}
\usepackage{amsmath}
\usepackage{bookmark}
\usepackage{ifthen}
%\usepackage{pgfpages}

\usetikzlibrary{calc}

\definecolor{logogreen1}{RGB}{52,166,77}
\definecolor{logogreen2}{RGB}{73,180,58}
\definecolor{logogreen3}{RGB}{117,200,44}
\definecolor{logogreen4}{RGB}{157,212,31}
\definecolor{logogreen5}{RGB}{67,149,55}
\definecolor{logogreen6}{RGB}{35,144,63}
\definecolor{titlegreen}{RGB}{72,147,65}
\definecolor{textgrey}{RGB}{138,140,142}
\definecolor{bulletsgreen}{RGB}{171,208,55}

\setbeamercolor*{structure}{fg=bulletsgreen}
\setbeamercolor*{palette primary}{fg=white,bg=logogreen3}
\setbeamercolor*{palette quaternary}{fg=white,bg=logogreen6}
\setbeamercolor*{normal text}{fg=textgrey, bg=white}
\setbeamercolor*{titlelike}{fg=titlegreen, bg=white}
\setbeamercolor*{block title}{fg=white, bg=logogreen1}

\makeatletter
\renewcommand{\@makefnmark}{\hbox{\textsuperscript{\tiny{\@thefnmark}}}}
\makeatother

\newenvironment<>{mydef}[1]{%
  \setbeamercolor{block title}{fg=white,bg=logogreen6}%
  \begin{block}#2{#1}}
{\end{block}}

\newenvironment<>{operations}
	{\begin{block}{Operations}{#1}}
	{\end{block}}


\setcounter{tocdepth}{4}
\newboolean{subsectiontoc}
\setboolean{subsectiontoc}{true}
\setbeamerfont{subsubsection in toc}{size=\tiny}
\setbeamertemplate{subsubsection in toc}{\leavevmode\leftskip=3em\inserttocsubsubsection\par}
%\addtobeamertemplate{footline}{\hfill\insertframenumber/\inserttotalframenumber\newline}{}

\AtBeginSection[]
{
  \begin{frame}
  \frametitle{Outline}
  \tiny{\tableofcontents[currentsection,subsubsectionstyle=hide]}
  \end{frame}
}
\AtBeginSubsection[]
{
	\ifthenelse{\boolean{subsectiontoc}}{
		\begin{frame}
			\frametitle{Outline}
			\tiny{\tableofcontents[sectionstyle=show/shaded,currentsubsection,subsubsectionstyle=hide]}
		\end{frame}
	}{
		\begin{frame}
			\frametitle{Outline}
			\tiny{\tableofcontents[sectionstyle=show/shaded,currentsubsection,subsubsectionstyle=show]}
		\end{frame}
	}
}

\newcommand{\deeptocsubsection}[1]{
  \setboolean{subsectiontoc}{false}
  \subsection{#1}
  \setboolean{subsectiontoc}{true}
}

\newcommand\AtPagemyUpperLeft[1]{\AtPageLowerLeft{%
\put(\LenToUnit{0mm},\LenToUnit{3.38mm}){#1}}}
\AddToShipoutPictureFG{
  \AtPagemyUpperLeft{{\includegraphics[width=.5cm,keepaspectratio]{r.png}}}
}


\title[{\makebox[.47\paperwidth]{Data Structures\hfill\insertframenumber/\inserttotalframenumber}}]{Data Structures}
\author{Maarten Dhondt}
\institute{Realdolmen}
\date{June 23, 2017}

%\setbeameroption{show notes on second screen=right}

\begin{document}

\frame{\titlepage}

\begin{frame}
	\frametitle{Who am I?}
	\begin{itemize}\setlength\itemsep{5mm}
		\item Master of Engineering: Computer Science (KUL)
		\begin{itemize}
			\item Computational informatics
		\end{itemize}
		\item Realdolmen: acADDemICT in 09/2015
		\item Current project: Planning infrastructure @ Infrabel
	\end{itemize}
\end{frame}

\begin{frame}
  \frametitle{Outline}
  \tiny{\tableofcontents[subsubsectionstyle=hide]}
\end{frame}

\section{Introductory Data Structures}

\begin{frame}
	\frametitle{What are Data Structures?}
	
	\begin{mydef}{Data Structure\footnote[frame]{\tiny{Encyclop\ae dia Britannica}}}
		A way in which data are stored for efficient search and retrieval. Different data structures are suited for different problems.
	\end{mydef}
	
	\vspace{5mm}
	
	\begin{itemize}
		\item $\text{Data type} \neq \text{data structure}$
		\item \texttt{java.util.HashSet} vs. hash table
		\item \texttt{array} vs. array
	\end{itemize}
\end{frame}

\subsection{Array}

\begin{frame}
	\frametitle{Array}
	
	\begin{definition}
		\begin{itemize}
			\item An indexed set of related elements.\footnote[frame]{\tiny{Oxford Dictionary}}
			\item An assemblage of items that are randomly accessible by integers, the index.\footnote[frame]{\tiny{National Institute of Standards \& Technology }}
		\end{itemize}
	\end{definition}
	\begin{itemize}
		\item Example: linear array
	\end{itemize}
	\begin{center}\begin{tikzpicture}[scale=0.7,every node/.style={minimum width=1cm,minimum height=1cm,transform shape,anchor=south west}]
		\draw (0,0) -- (3,0) (0,1) -- (3,1) (0,0) -- (0,1) (1,0) -- (1,1) (2,0) -- (2,1) (5,1) -- (7,1) -- (7,0) -- (5,0) (6,0) -- (6,1) (7,0) -- (7,1);
		\draw[dashed] (3,0) -- (3,1) -- (5,1) -- (5,0) -- cycle;
		\node at (0,-1) {$0$};
		\node at (1,-1) {$1$};
		\node at (2,-1) {$2$};
		\node at (5,-1) {$n-1$};
		\node at (6,-1) {$n$};
		\node[minimum width=2cm] at (3,-1) {$\cdots$};
	\end{tikzpicture}\end{center}
\end{frame}

\begin{frame}
	\frametitle{Array}
	
	\begin{operations}
	\begin{itemize}
		\only<1-3>{\item \parbox{2cm}{\texttt{get}} \phantom{$O($}}
		\only<4->{\item \parbox{2cm}{\texttt{get}} $O(1)$}
		\only<1-6>{\item \parbox{2cm}{\texttt{set}} \phantom{$O($}}
		\only<7->{\item \parbox{2cm}{\texttt{set}} $O(1)$}
		\only<1-13>{\item \parbox{2cm}{\texttt{indexOf}} \phantom{$O($}}
		\only<14->{\item \parbox{2cm}{\texttt{indexOf}} $O(n)$}
	\end{itemize}
	\end{operations}
	\begin{center}
		\begin{tikzpicture}[scale=0.7,every node/.style={minimum width=1cm,minimum height=1cm,transform shape,anchor=south west}]
			\draw (0,0) -- (3,0) (0,1) -- (3,1) (0,0) -- (0,1) (1,0) -- (1,1) (2,0) -- (2,1) (5,1) -- (7,1) -- (7,0) -- (5,0) (6,0) -- (6,1) (7,0) -- (7,1);
			\draw[dashed] (3,0) -- (3,1) -- (5,1) -- (5,0) -- cycle;
			\draw[white] (-.05,0) -- (-.05,1.05);
			\only<3>{\node[draw,thick,red] at (.985,-.015) {\ldots};}
			\only<6>{\node[draw,thick,red] at (1.985,-.015) {\ldots};}
			\only<9>{\node[draw,thick,red] at (-.015,-.015) {\ldots};}
			\only<10>{\node[draw,thick,red] at (.985,-.015) {\ldots};}
			\only<11>{\node[draw,thick,red] at (1.985,-.015) {\ldots};}
			\only<12>{\node[draw,thick,red] at (4.985,-.015) {\ldots};}
			\only<13>{\node[draw,thick,red] at (5.985,-.015) {\ldots};}
			\node at (0,-1) {$0$};
			\node at (1,-1) {$1$};
			\node at (2,-1) {$2$};
			\node at (5,-1) {$n-1$};
			\node at (6,-1) {$n$};
			\node[minimum width=2cm] at (3,-1) {$\cdots$};
		\end{tikzpicture}
	\end{center}
	\begin{center}
		\only<1>{\ \vphantom{\texttt{gO(}}}
		\only<2-4>{\texttt{get(1)} \vphantom{\texttt{gO(}}}
		\only<5-7>{\texttt{set(2)} \vphantom{\texttt{gO(}}}
		\only<8-13>{\texttt{indexOf(object)} \vphantom{\texttt{gO(}}}
		\only<14>{\ \vphantom{\texttt{gO(}}}
	\end{center}
	
\end{frame}

\subsection{Linked List}

\begin{frame}
	\frametitle{Linked List}
	
	\begin{definition}
		A linked list is a data structure in which the objects are arranged in a linear order. Unlike arrays in which the linear order is determined by indices, the order is determined by a pointer in each object.\footnote[frame]{\tiny{Introduction to Algorithms By Cormen, Leierson, Rivest \& Stein}}
	\end{definition}
	
	\begin{itemize}
		\item Different types: singly, doubly, multiply, circular, \ldots
		\item Example: doubly linked list
	\end{itemize}
	
	\begin{center}
		\begin{tikzpicture}[scale=0.5,every node/.style={minimum width=1cm,minimum height=1cm,transform shape,anchor=south west,draw}]
			\node[draw=none] (head) at (0,0) {Head};
			\node (firstl) at (2,0) {};
			\node (firstm) at (3,0) {$\text{O}_1$};
			\node (firstr) at (4,0) {};
			\node (secondl) at (6,0) {};
			\node (secondm) at (7,0) {$\text{O}_2$};
			\node (secondr) at (8,0) {};
			\node (thirdl) at (10,0) {};
			\node (thirdm) at (11,0) {$\text{O}_3$};
			\node (thirdr) at (12,0) {};
			\node[draw=none] (tail) at (14,0) {Tail};
			\draw[>=latex,->,OliveGreen] (head.east) -- ($(firstl.west)+(.5,0)$);
			\draw[>=latex,->,OliveGreen] ($(firstr.east)+(-.5,.2)$) -- ($(secondl.west)+(.5,.2)$);
			\draw[>=latex,<-,red] ($(firstr.east)+(-.5,-.2)$) -- ($(secondl.west)+(.5,-.2)$);
			\draw[>=latex,->,OliveGreen] ($(secondr.east)+(-.5,.2)$) -- ($(thirdl.west)+(.5,.2)$);
			\draw[>=latex,<-,red] ($(secondr.east)+(-.5,-.2)$) -- ($(thirdl.west)+(.5,-.2)$);
			\draw[>=latex,->,red] (tail.west) -- ($(thirdr.east)-(.5,0)$);
		\end{tikzpicture}
	\end{center}
\end{frame}

\begin{frame}
	\frametitle{Linked List}
	
	\begin{operations}
	\begin{itemize}
		\only<1-4>{\item \parbox{5cm}{\texttt{add/remove first/last}} \vphantom{$O($}}
		\only<5->{\item \parbox{5cm}{\texttt{add/remove first/last}} $O(1)$}
		\only<1-8>{\item \parbox{5cm}{\texttt{get/insertAt}} \vphantom{$O($}}
		\only<9->{\item \parbox{5cm}{\texttt{get/insertAt}} $O(n)$}
		\only<1-12>{\item \parbox{5cm}{\texttt{indexOf}} \vphantom{$O($}}
		\only<13->{\item \parbox{5cm}{\texttt{indexOf}} $O(n)$}
	\end{itemize}
	\end{operations}
	\begin{center}
		\begin{tikzpicture}[scale=0.5,every node/.style={minimum width=1cm,minimum height=1cm,transform shape,anchor=south west,draw}]
			\draw[white] (0,-1.55) -- (0,1.05);
			\node[draw=none] (head) at (0,0) {Head};
			\node (firstl) at (2,0) {};
			\node (firstm) at (3,0) {$\text{O}_1$};
			\node (firstr) at (4,0) {};
			\node (secondl) at (6,0) {};
			\node (secondm) at (7,0) {$\text{O}_2$};
			\node (secondr) at (8,0) {};
			\node (thirdl) at (10,0) {};
			\node (thirdm) at (11,0) {$\text{O}_3$};
			\node (thirdr) at (12,0) {};
			\node[draw=none] (tail) at (14,0) {Tail};
			\only<1-3,6->{\draw[>=latex,->,OliveGreen] (head.east) -- ($(firstl.west)+(.5,0)$);}
			\draw[>=latex,->,OliveGreen] ($(firstr.east)+(-.5,.2)$) -- ($(secondl.west)+(.5,.2)$);
			\draw[>=latex,<-,red] ($(firstr.east)+(-.5,-.2)$) -- ($(secondl.west)+(.5,-.2)$);
			\only<1-7,10->{\draw[>=latex,->,OliveGreen] ($(secondr.east)+(-.5,.2)$) -- ($(thirdl.west)+(.5,.2)$);}
			\only<1-7,10->{\draw[>=latex,<-,red] ($(secondr.east)+(-.5,-.2)$) -- ($(thirdl.west)+(.5,-.2)$);}
			\draw[>=latex,->,red] (tail.west) -- ($(thirdr.east)-(.5,0)$);
			\only<3-5>{
				\node (newl) at (.5,-1.5) {};
				\node (newm) at (1.5,-1.5) {$\text{O}_4$};
				\node (newr) at (2.5,-1.5) {};
			}
			\only<4-5>{
				\draw[>=latex,->,OliveGreen] (head.east) -- ($(newl.west)+(.5,0)$);
				\draw[>=latex,->,OliveGreen] ($(newr.east)+(-.5,.2)$) -- ($(firstl.west)+(.5,.2)$);
				\draw[>=latex,<-,red] ($(newr.east)+(-.5,-.2)$) -- ($(firstl.west)+(.5,-.2)$);
			}
			\only<7-9>{
				\node (newl) at (8,-1.5) {};
				\node (newm) at (9,-1.5) {$\text{O}_4$};
				\node (newr) at (10,-1.5) {};
			}
			\only<8-9>{
				\draw[>=latex,->,OliveGreen] ($(secondr.east)+(-.7,0)$) -- ($(newl.west)+(.3,0)$);
				\draw[>=latex,<-,red] ($(secondr.east)+(-.3,0)$) -- ($(newl.west)+(.7,0)$);
				\draw[>=latex,<-,OliveGreen] ($(thirdl.west)+(.3,0)$) -- ($(newr.east)+(-.7,0)$);
				\draw[>=latex,->,red] ($(thirdl.west)+(.7,0)$) -- ($(newr.east)+(-.3,0)$);
			}
			\only<11>{\node[draw,thick,red] at (2.985,-.015) {};}
			\only<12>{\node[draw,thick,red] at (6.985,-.015) {};}
		\end{tikzpicture}
	\end{center}
	\begin{center}
		\only<1>{\ \vphantom{\texttt{gAO(/}}}
		\only<2-5>{\texttt{addFirst(O$_4$)} \vphantom{\texttt{gAO(/}}}
		\only<6-9>{\texttt{insertAt(2)} \vphantom{\texttt{gAO(/}}}
		\only<10-13>{\texttt{indexOf(O$_2$)} \vphantom{\texttt{gAO(/}}}
		\only<14>{\ \vphantom{\texttt{gAO(/}}}
	\end{center}
\end{frame}

\subsection{Hash Table}

\begin{frame}
	\frametitle{Hash Table}
	
	\begin{definition}
		A dictionary in which keys are mapped to array positions by hash functions.\footnote[frame]{\tiny{National Institute of Standards \& Technology }}
	\end{definition}
	
	\begin{itemize}
		\item Hash functions: determinism, uniformity, defined range, data normalisation, non-invertible, perfect, \ldots
		\item Collisions resolution: chaining, open addressing, \ldots
		\item Example:
	\end{itemize}
	
	\begin{center}
		\vspace*{-7mm}
		\begin{tikzpicture}[scale=0.4,every node/.style={minimum width=2cm,minimum height=6mm,transform shape,anchor=south west,draw}]
			\node[draw=none] (key) at (0,4.8) {Keys};
			\node[draw=none] (hash1) at (3,5.2) {Hash};
			\node[draw=none] (hash2) at (3,4.8) {function};
			\node[draw=none] (buckets) at (6.6,4.8) {Buckets};
			\node (k1) at (0,3.3) {$k_1$};
			\node (k2) at (0,2.1) {$k_2$};
			\node (k3) at (0,0.9) {$k_3$};
			\node[draw=none,fill=OliveGreen!10,minimum height=48mm] at (3,0) {};
			\node[draw=none,minimum width=5mm] (b0) at (6,4.2) {$00$};
			\node[draw=none,minimum width=5mm] (b1) at (6,3.6) {$01$};
			\node[draw=none,minimum width=5mm] (b2) at (6,3) {$02$};
			\node[draw=none,minimum width=5mm] (b3) at (6,2.4) {$03$};
			\node[draw=none,minimum width=5mm] (b4) at (6.1,1.8) {$\vdots$};
			\node[draw=none,minimum width=5mm] (b13) at (6,1.2) {$13$};
			\node[draw=none,minimum width=5mm] (b14) at (6,.6) {$14$};
			\node[draw=none,minimum width=5mm] (b15) at (6,0) {$15$};
			\node at (6.6,4.2) {};
			\node at (6.6,3.6) {$v_2$};
			\node at (6.6,3) {$v_1$};
			\node at (6.6,2.4) {};
			\node[draw=none] at (6.6,1.8) {$\vdots$};
			\node at (6.6,1.2) {};
			\node at (6.6,.6) {$v_3$};
			\node at (6.6,0) {};
			\draw[>=latex,->] (k1.east) -- ($(k1.east)+(1.2,0)$) -- ($(b2.west)-(1.2,0)$) -- (b2.west);
			\draw[>=latex,->] (k2.east) -- ($(k2.east)+(1.2,0)$) -- ($(b1.west)-(1.2,0)$) -- (b1.west);
			\draw[>=latex,->] (k3.east) -- ($(k3.east)+(1.2,0)$) -- ($(b14.west)-(1.2,0)$) -- (b14.west);
		\end{tikzpicture}
	\end{center}
\end{frame}

\begin{frame}
	\frametitle{Hash Table}
	
	\begin{operations}\begin{itemize}
		\only<1-4>{\item \parbox{5cm}{\texttt{put}} \vphantom{$O($}}
		\only<5->{\item \parbox{5cm}{\texttt{put}} $O(1)\,\,/\, O(n)$}
		\only<1-8>{\item \parbox{5cm}{\texttt{remove}} \vphantom{$O($}}
		\only<9->{\item \parbox{5cm}{\texttt{remove}} $O(1)\,\,/\, O(n)$}
		\only<1-12>{\item \parbox{5cm}{\texttt{get}} \vphantom{$O($}}
		\only<13->{\item \parbox{5cm}{\texttt{get}} $O(1)\,\,/\, O(n)$}
	\end{itemize}\end{operations}
	
	\begin{center}
		\begin{tikzpicture}[scale=0.5,every node/.style={minimum width=2cm,minimum height=6mm,transform shape,anchor=south west,draw}]
			\draw[white] (-.6,-.1) -- (8.7,-.1);
			\node[draw=none] (key) at (0,4.8) {Keys};
			\node[draw=none] (hash1) at (3,5.2) {Hash};
			\node[draw=none] (hash2) at (3,4.8) {function};
			\node[draw=none] (buckets) at (6.6,4.8) {Buckets};
			\node (k1) at (0,3.3) {$k_1$};
			\only<1-6,10,14>{\node (k2) at (0,2.1) {$k_2$};}
			\only<7> {\node[text=red] (k2) at (0,2.1) {$k_2$};}
			\only<8-9> {\node[text=red!50] (k2) at (0,2.1) {$k_2$};}
			\only<11-13> {\node[text=red,draw=red] (k2) at (0,2.1) {$k_2$};}
			\node (k3) at (0,0.9) {$k_3$};
			\only<3-5>{\node (k4) at (-.5,1.5) {$k_4$};}
			\node[draw=none,fill=OliveGreen!10,minimum height=48mm] at (3,0) {};
			\node[draw=none,minimum width=5mm] (b0) at (6,4.2) {$00$};
			\node[draw=none,minimum width=5mm] (b1) at (6,3.6) {$01$};
			\node[draw=none,minimum width=5mm] (b2) at (6,3) {$02$};
			\node[draw=none,minimum width=5mm] (b3) at (6,2.4) {$03$};
			\node[draw=none,minimum width=5mm] (b4) at (6.1,1.8) {$\vdots$};
			\node[draw=none,minimum width=5mm] (b13) at (6,1.2) {$13$};
			\node[draw=none,minimum width=5mm] (b14) at (6,.6) {$14$};
			\node[draw=none,minimum width=5mm] (b15) at (6,0) {$15$};
			\node at (6.6,4.2) {};
			\only<4-5>{\node[text=red,draw=none] at (6.6,4.2) {$v_4$};}
			\only<1-7,10-11,14>{\node at (6.6,3.6) {$v_2$};}
			\only<8-9>{\node[text=red!50] at (6.6,3.6) {$v_2$};}
			\only<12-13>{\node[text=red] at (6.6,3.6) {$v_2$};}
			\node at (6.6,3) {$v_1$};
			\node at (6.6,2.4) {};
			\node[draw=none] at (6.6,1.8) {$\vdots$};
			\node at (6.6,1.2) {};
			\node at (6.6,.6) {$v_3$};
			\node at (6.6,0) {};
			\draw[>=latex,->] (k1.east) -- ($(k1.east)+(1.2,0)$) -- ($(b2.west)-(1.2,0)$) -- (b2.west);
			\only<1-7,10-11,14>{\draw[>=latex,->] (k2.east) -- ($(k2.east)+(1.2,0)$) -- ($(b1.west)-(1.2,0)$) -- (b1.west);}
			\only<8-9>{\draw[>=latex,->,red!50] (k2.east) -- ($(k2.east)+(1.2,0)$) -- ($(b1.west)-(1.2,0)$) -- (b1.west);}
			\only<12-13>{\draw[>=latex,->,red] (k2.east) -- ($(k2.east)+(1.2,0)$) -- ($(b1.west)-(1.2,0)$) -- (b1.west);}
			\draw[>=latex,->] (k3.east) -- ($(k3.east)+(1.2,0)$) -- ($(b14.west)-(1.2,0)$) -- (b14.west);
			\only<4-5>{\draw[>=latex,->,red] (k4.east) -- ($(k4.east)+(1.7,0)$) -- ($(b0.west)-(1.2,0)$) -- (b0.west);}
		\end{tikzpicture}
	\end{center}
	
	\begin{center}
		\only<1>{\ \vphantom{\texttt{g(/}}}
		\only<2-5>{\texttt{put(O$_4$)} \vphantom{\texttt{g(/}}}
		\only<6-9>{\texttt{remove(O$_2$)} \vphantom{\texttt{g(/}}}
		\only<10-13>{\texttt{get(O$_2$)} \vphantom{\texttt{g(/}}}
		\only<14>{\ \vphantom{\texttt{g(/}}}
	\end{center}
\end{frame}

\deeptocsubsection{Tree}

\begin{frame}
	\frametitle{Tree}
	\ldots
\end{frame}

\subsubsection{Heap}

\begin{frame}
	\frametitle{Binary Heap}
	\ldots
\end{frame}
\begin{frame}
	\frametitle{Binary Min-Heap}
	\ldots
\end{frame}

\subsubsection{Red-Black Tree}

\begin{frame}
	\frametitle{Red-Black Tree}
	\ldots
\end{frame}
\begin{frame}
	\frametitle{Red-Black Tree}
	\ldots
\end{frame}

\section{Java Collection API \& Map API}

\subsection{Java Collection API}

\begin{frame}
	\frametitle{Java Collection API}
	\ldots
\end{frame}
\begin{frame}
	\frametitle{Java Collection API}
	\ldots
\end{frame}

\subsection{Java Map API}

\begin{frame}
	\frametitle{Java Map API}
	\ldots
\end{frame}
\begin{frame}
	\frametitle{Java Map API}
	\ldots
\end{frame}


\section{Advanced Data Structures}

\subsection{Stuff\ldots}

\begin{frame}
	\frametitle{Stuff\ldots}
	\ldots
\end{frame}

\end{document}