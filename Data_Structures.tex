\documentclass{beamer}
\usetheme{Copenhagen}
\useinnertheme{circles}
\useoutertheme{split}
\setbeamertemplate{itemize items}[triangle]
\beamertemplatenavigationsymbolsempty

\usepackage[english]{babel}
\usepackage[utf8]{inputenc}
\usepackage[T1]{fontenc}
\usepackage{lmodern}
\usepackage{textcomp}
\usepackage{tikz}
\usepackage{eso-pic}
\usepackage{amsmath}
\usepackage{bookmark}
%\usepackage{pgfpages}

\definecolor{logogreen1}{RGB}{52,166,77}
\definecolor{logogreen2}{RGB}{73,180,58}
\definecolor{logogreen3}{RGB}{117,200,44}
\definecolor{logogreen4}{RGB}{157,212,31}
\definecolor{logogreen5}{RGB}{67,149,55}
\definecolor{logogreen6}{RGB}{35,144,63}
\definecolor{titlegreen}{RGB}{72,147,65}
\definecolor{textgrey}{RGB}{138,140,142}
\definecolor{bulletsgreen}{RGB}{171,208,55}

\setbeamercolor*{structure}{fg=bulletsgreen}
\setbeamercolor*{palette primary}{fg=white,bg=logogreen3}
\setbeamercolor*{palette quaternary}{fg=white,bg=logogreen6}
\setbeamercolor*{normal text}{fg=textgrey, bg=white}
\setbeamercolor*{titlelike}{fg=titlegreen, bg=white}
\setbeamercolor*{block title}{fg=white, bg=logogreen1}

\makeatletter
\renewcommand{\@makefnmark}{\hbox{\textsuperscript{\tiny{\@thefnmark}}}}
\makeatother

\newenvironment<>{mydef}[1]{%
  \setbeamercolor{block title}{fg=white,bg=logogreen6}%
  \begin{block}#2{#1}}
{\end{block}}

\newenvironment<>{operations}
	{\begin{block}{Operations}{#1}}
	{\end{block}}

\AtBeginSection[]
{
  \begin{frame}
  \frametitle{Outline}
  \tiny{\tableofcontents[currentsection]}
  \end{frame}
}
\AtBeginSubsection[]
{
  \begin{frame}
  \frametitle{Outline}
  \tiny{\tableofcontents[sectionstyle=show/shaded,currentsubsection]}
  \end{frame}
}

\title{Data Structures}
\author{Maarten Dhondt}
\institute{Realdolmen}
\date{June 23, 2017}
\newcommand\AtPagemyUpperLeft[1]{\AtPageLowerLeft{%
\put(\LenToUnit{0mm},\LenToUnit{3.38mm}){#1}}}
\AddToShipoutPictureFG{
  \AtPagemyUpperLeft{{\includegraphics[width=.5cm,keepaspectratio]{r.png}}}
}%

%\setbeameroption{show notes on second screen=right}

\begin{document}

%\section*{Data Structures}
%\hypertarget{datastructures}{}
%\bookmark[dest=datastructures]{Data Structures}

\frame{\titlepage}

\begin{frame}
	\frametitle{Who am I?}
	\begin{itemize}\setlength\itemsep{5mm}
		\item Master of Engineering: Computer Science (KUL)
		\begin{itemize}
			\item Computational informatics
		\end{itemize}
		\item Realdolmen: acADDemICT in 09/2015
		\item Current project: Planning infrastructure @ Infrabel
	\end{itemize}
\end{frame}

\begin{frame}
  \frametitle{Outline}
  \tiny{\tableofcontents}
\end{frame}

\section{Introductory Data Structures}

\begin{frame}
	\frametitle{What are Data Structures?}
	
	\begin{mydef}{Data Structure\footnote[frame]{\tiny{Encyclop\ae dia Britannica}}}
		A way in which data are stored for efficient search and retrieval. Different data structures are suited for different problems.
	\end{mydef}
	
	\vspace{5mm}
	
	\begin{itemize}
		\item $\text{Data type} \neq \text{data structure}$
		\item \texttt{java.util.HashSet} vs. hash table
		\item \texttt{array} vs. array
	\end{itemize}
\end{frame}

\subsection{Array}

\begin{frame}
	\frametitle{Array}
	
	\begin{definition}
		\begin{itemize}
			\item An indexed set of related elements\footnote[frame]{\tiny{Oxford Dictionary}}
			\item An assemblage of items that are randomly accessible by integers, the index\footnote[frame]{\tiny{National Institute of Standards \& Technology }}
		\end{itemize}
	\end{definition}
	\begin{itemize}
		\item Example: linear array
	\end{itemize}
	\begin{center}\begin{tikzpicture}[scale=0.7,every node/.style={minimum width=1cm,minimum height=1cm,transform shape,anchor=south west}]
		\draw (0,0) -- (3,0) (0,1) -- (3,1) (0,0) -- (0,1) (1,0) -- (1,1) (2,0) -- (2,1) (5,1) -- (7,1) -- (7,0) -- (5,0) (6,0) -- (6,1) (7,0) -- (7,1);
		\draw[dashed] (3,0) -- (3,1) -- (5,1) -- (5,0) -- cycle;
		\node at (0,-1) {$0$};
		\node at (1,-1) {$1$};
		\node at (2,-1) {$2$};
		\node at (5,-1) {$n-1$};
		\node at (6,-1) {$n$};
		\node[minimum width=2cm] at (3,-1) {$\cdots$};
	\end{tikzpicture}\end{center}
\end{frame}

\begin{frame}
	\frametitle{Array}
	
	\begin{operations}
	\begin{itemize}
		\only<1-3>{\item \parbox{2cm}{\texttt{get}} \phantom{$O($}}
		\only<4->{\item \parbox{2cm}{\texttt{get}} $O(1)$}
		\only<1-6>{\item \parbox{2cm}{\texttt{set}} \phantom{$O($}}
		\only<7->{\item \parbox{2cm}{\texttt{set}} $O(1)$}
		\only<1-13>{\item \parbox{2cm}{\texttt{indexOf}} \phantom{$O($}}
		\only<14->{\item \parbox{2cm}{\texttt{indexOf}} $O(n)$}
	\end{itemize}
	\end{operations}
	\begin{center}
		\begin{tikzpicture}[scale=0.7,every node/.style={minimum width=1cm,minimum height=1cm,transform shape,anchor=south west}]
			\draw (0,0) -- (3,0) (0,1) -- (3,1) (0,0) -- (0,1) (1,0) -- (1,1) (2,0) -- (2,1) (5,1) -- (7,1) -- (7,0) -- (5,0) (6,0) -- (6,1) (7,0) -- (7,1);
			\draw[dashed] (3,0) -- (3,1) -- (5,1) -- (5,0) -- cycle;
			\draw[white] (-.05,0) -- (-.05,1.05);
			\only<3>{\node[draw,thick,red] at (.985,-.015) {\ldots};}
			\only<6>{\node[draw,thick,red] at (1.985,-.015) {\ldots};}
			\only<9>{\node[draw,thick,red] at (-.015,-.015) {\ldots};}
			\only<10>{\node[draw,thick,red] at (.985,-.015) {\ldots};}
			\only<11>{\node[draw,thick,red] at (1.985,-.015) {\ldots};}
			\only<12>{\node[draw,thick,red] at (4.985,-.015) {\ldots};}
			\only<13>{\node[draw,thick,red] at (5.985,-.015) {\ldots};}
			\node at (0,-1) {$0$};
			\node at (1,-1) {$1$};
			\node at (2,-1) {$2$};
			\node at (5,-1) {$n-1$};
			\node at (6,-1) {$n$};
			\node[minimum width=2cm] at (3,-1) {$\cdots$};
		\end{tikzpicture}
	\end{center}
	\begin{center}
		\only<1>{\ \vphantom{\texttt{gO(}}}
		\only<2-4>{\texttt{get(1)} \vphantom{\texttt{gO(}}}
		\only<5-7>{\texttt{set(2)} \vphantom{\texttt{gO(}}}
		\only<8-13>{\texttt{indexOf(object)} \vphantom{\texttt{gO(}}}
		\only<14>{\ \vphantom{\texttt{gO(}}}
	\end{center}
	
\end{frame}

\subsection{Linked List}

\begin{frame}
	\frametitle{Linked List}
	\ldots
\end{frame}
\begin{frame}
	\frametitle{Linked List}
	\ldots
\end{frame}

\subsection{Hash Table}

\begin{frame}
	\frametitle{Hash Table}
	\ldots
\end{frame}
\begin{frame}
	\frametitle{Hash Table}
	\ldots
\end{frame}

\subsection{Heap}

\begin{frame}
	\frametitle{Tree}
	\ldots
\end{frame}

\begin{frame}
	\frametitle{Binary Heap}
	\ldots
\end{frame}
\begin{frame}
	\frametitle{Binary Min-Heap}
	\ldots
\end{frame}

\subsection{Red-Black Tree}

\begin{frame}
	\frametitle{Red-Black Tree}
	\ldots
\end{frame}
\begin{frame}
	\frametitle{Red-Black Tree}
	\ldots
\end{frame}

\section{Java Collection API \& Map API}

\subsection{Java Collection API}

\begin{frame}
	\frametitle{Java Collection API}
	\ldots
\end{frame}
\begin{frame}
	\frametitle{Java Collection API}
	\ldots
\end{frame}

\subsection{Java Map API}

\begin{frame}
	\frametitle{Java Map API}
	\ldots
\end{frame}
\begin{frame}
	\frametitle{Java Map API}
	\ldots
\end{frame}


\section{Advanced Data Structures}

\subsection{Stuff\ldots}

\begin{frame}
	\frametitle{Stuff\ldots}
	\ldots
\end{frame}

\end{document}